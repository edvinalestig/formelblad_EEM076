\subsection{Kretsar}
\subsubsection*{Node voltage analysis}
Analysera spänningsskillnader gentemot en referensnod (jord eller den nod med flest kopplingar). Lös med ekvationssystem.

\begin{enumerate}
	\item Välj en referensnod och sätt den till 0 V.
	\item Sätt variabler för varje nod.
	\item Applicera KCL på varje nod.
	\item Räkna ut spänningen genom att räkna ut spänningsdifferensen mellan två noder.
\end{enumerate}

Tips: Räkna $ I_{out} $ som positiv i varje resistor. 

\subsubsection*{Supernod}
Spänningskälla som ej är direkt kopplad till referensnoden kan göras om till en supernod. Nodens spänning är källans spänning och båda ändars kopplingar räknas som supernodens kopplingar.

\subsubsection*{Mesh current analysis}
Analysera loopar i en krets (medsols). Applicera KVL på varje loop. Lös med ekvationssystem.

\subsubsection*{Supermesh}
Strömkälla i kretsen. Kombinera loopar in i en större superloop. $ I_{super} = I_1 - I_2 $

\subsubsection*{Superposition}
Går endast att applicera på linjära kretsar med flera ström- och/eller spänningskällor. Varje källa kan analyseras separat för att sedan läggas ihop.
\begin{enumerate}
	\item Stäng av alla källor förutom en.
	\begin{itemize}
		\item v = 0 blir en kortsluten krets.
		\item I = 0 blir en öppen krets.
		\item Räkna ut källans kretspåverkan.
	\end{itemize}
	\item Lägg ihop alla källors påverkan.
\end{enumerate}