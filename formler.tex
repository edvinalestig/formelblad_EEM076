\documentclass{article}
\usepackage{tabto}
\usepackage{graphicx}
\usepackage{float}
\usepackage{amsmath}
\usepackage{esint}

\renewcommand{\baselinestretch}{1.25}

\title{Formelblad elektriska kretsar och fält EEM076}
\author{Edvin Alestig}

\begin{document}
\maketitle

\section{Storheter och enheter}

\textbf{Storhet} \tab \textbf{Enhet}
\\
Kraft (F) \tab Newton (N)
\\
Laddning (Q)  \tab Coloumb (C)
\\
Spänning (v) \tab Volt (V)
\\
Ström (I) \tab Ampere (A)
\\
Resistans (R) \tab Ohm ($\Omega$)
\\
Effekt (P) \tab Watt (W)
\\
Energi (W) \tab Joule (J)
\\
Kapacitans (C) \tab Farad (F)
\\
Induktans (L) \tab Henry (H)
\\
Elektriskt fält (E) \tab Newton/Coulomb (N/C)

%% -- %%
\section{Lagar}

\textbf{Ohms lag} \tab  $ v=RI $
\\
\textbf{Effektlagen} \tab  $ P = Iv = RI^2 = \frac{v^2}{R} $
\\
\textbf{Kirchhoffs spänningslag (KVL)} \tab $ \sum v = 0 $  i en loop
\\
\textbf{Kirchhoffs strömlag (KCL)} \tab $ \sum I_{in} = \sum I_{out} $  i en nod
\\
\textbf{Energiprincipen} \tab $ \sum P = 0 $ i en krets
\\
\textbf{Coloumbs lag} \tab \( \vec{F}_{12} = k_e \frac{q_1q_2}{r^2} \hat{r}_{12} \)
\\
\textbf{Gauss lag} \tab \( \vec{\Phi} = \oiint \vec{E} \cdot d\vec{A} = \frac{Q}{\varepsilon_0} \)

\section{Konstanter}

Coloumbkonstanten \tab \( k_e = \frac{1}{4 \pi \varepsilon_0} \)
\\
Elektrisk permittivitet i vakuum \tab \( \varepsilon_0 = \frac{10^{-9}}{36 \pi} \)
\\
Elementarladdningen \tab \( e = 1.602 \cdot 10^{-19} \) C

%% -- %%
\section{Formler}
\subsection{Kretsar}
\[ I(t) = \frac{dq(t)}{dt} \]

\[ Q(t) = \int_{t_0}^t I(t) \cdot dt + Q(t_0) \]

\[ W = \int_{t_1}^{t_2} P(t) \cdot dt \]

I kondensatorer:
\[ Q = Cv \]
\[ I = \frac{dQ}{dt} = C \frac{dv}{dt} \]
\[ P = IV = Cv \frac{dv}{dt} \]
\[ W = \int_{t_0}^t P(t) \cdot dt = \int_{t_0}^t Cv \frac{dv}{dt} = C \int_{v(t_0)}^{v(t)} v \cdot dv = \frac{C}{2}(v(t)^2 - v(t_0)^2) \]
\[ W = \frac{Cv^2}{2}, v(t_0) = 0 \]
\[ v(t) = \frac{1}{C} \int_{t_0}^t I(t)\cdot dt + v(t_0) \]

I induktorer:
\[ v = L \frac{dI}{dt} \]
\[ W = \frac{LI^2}{2}, I(t_0) = 0 \]
\[ I(t) = \frac{1}{L} \int_{t_0}^t v(t) \cdot dt + I(t_0) \]
\[ P = IV = LI \frac{di}{dt} \]

\subsection{Elektriska fält}
\[ \vec{E} = k_e \frac{q}{r^2} \hat{r}_{12} \]
\[ \vec{F}_E  = q \vec{E} \]
\[ \vec{E}_{total} = \sum \vec{E}_i \]
\[ \vec{E}_{total} = \int_{L1}^{L2} \vec{E}_l \cdot dl \]

Flytta laddningar:
\[ W =\int_R^{\infty} \vec{F} \cdot d\vec{r} = \frac{-k_eq_1q_2}{R} \textrm{ (utanför fält)} \]
\[ W = -qE_0r \]

Dipoler:
\[ \vec{P} = q \vec{d} \]
\[ \vec{\tau} = \vec{p} \times \vec{E} \]
\[ \tau = p \cdot E \cdot \sin{\theta} \]

Elektriskt flöde (flux):
\[ \vec{\Phi} = \sum \vec{E}_i \]
\[ \vec{\Phi} = \int_{L1}^{L2} \vec{E}_l \cdot d\vec{l} \]

Två dimensioner: 
\[ \vec{\Phi} = \iint \vec{E}\hat{n} \cdot d\vec{A} = \iint \vec{E} \cdot d\vec{A} \cdot \cos{\theta} \]

Elektrisk potential:
\[ \frac{W}{q} = -\int_A^B \vec{E} \cdot d\vec{r} = \Delta V \]
\[ \vec{E} = -\frac{dV}{d\vec{r}} = \nabla v  = grad(v) \textrm{ (typ flerdimensionell derivata)} \]

%% -- %%
\section{Ekvivalenta kretsar}
\subsection{Seriekoppling}

Resistans $ R_{eq} = \sum R_n $
\\
Kapacitans  $ C_ {eq} = (\sum C_n ^{-1})^{-1} $ \tab (\( C_{eq} = \frac{C_1C_2}{C_1+C_2} \) vid endast 2 kondensatorer)
\\
Induktans $ L_{eq} = \sum L_n $
\\
Impedans \( Z_{eq} = \sum Z_n \)
\\ \\
Spänningsdelning \tab $ v_{n} = R_nI = \frac{R_n}{R_{eq}} \cdot v_{total} $

\subsection{Parallellkoppling}

Resistans $ R_{eq} = (\sum R_n^{-1})^{-1} $ \tab ($ R_{eq} = \frac{R_1R_2}{R_1 + R_2} $ vid endast 2 resistorer)
\\
Kapacitans $ C_{eq} = \sum C_n $
\\
Induktans $ L_{eq} = (\sum L_n^{-1})^{-1} $ \tab (\(L_{eq} = \frac{L_1L_2}{L_1+L_2}\) vid endast 2 induktorer)
\\
Impedans \( Z_{eq} = (\sum Z_n^{-1})^{-1} \) \tab (\( Z_{eq} = \frac{Z_1Z_2}{Z_1 + Z_2} \) vid endast 2 impedanser)
\\ \\
Strömdelning \tab $ I_1 = \frac{R_2}{R_1+R_2} \cdot I_{total} \quad I_2 = \frac{R_1}{R_1+R_2} \cdot I_{total} $

\subsection{Thévenin equivalent circuit (behöver förbättras)}

\begin{figure}[H]
    \centering
        \includegraphics[scale=0.5]{thevenin.png}
    \label{fig:thevenin}
\end{figure}

\begin{enumerate}
    \item Disconnect the load \(R_L\) and replace with an open circuit.
    \item Find the open circuit voltage \(V_{oc}\).
    \item Find the equivalent resistance \(R_{eq}\) of the network with all independent sources turned off.
    \item \(v_{th} = v_{oc}\) and \(R_{th} = R_{eq}\).
\end{enumerate}

\subsection{Norton equivalent circuit (behöver förbättras)}

\begin{figure}[H]
    \centering
        \includegraphics[scale=0.5]{norton.png}
    \label{fig:norton}
\end{figure}

\begin{enumerate}
    \item Replace the load \(R_L\) with a short circuit.
    \item Find the short circuit current \(I_{sc}\).
    \item Find the equivalent resistance \(R_{eq}\) of the network with all independent sources turned off.
    \item \(I_N = I_{sc}\) and \(R_N = R_{eq}\).
\end{enumerate}

\subsection{Source transformation - Thévenin and Norton}
\(R_{th} = R_N = R_{eq}\) and \(v_{th} = I_NR_{eq}\)
\\
Genom att kombinera Thévenin och Norton kan man kraftigt förenkla en delkrets.


%% -- %%
\section{Verktyg och metoder}
\subsection{Kretsar}
\subsubsection*{Node voltage analysis}
Analysera spänningsskillnader gentemot en referensnod (jord eller den nod med flest kopplingar). Lös med ekvationssystem.

\begin{enumerate}
	\item Välj en referensnod och sätt den till 0 V.
	\item Sätt variabler för varje nod.
	\item Applicera KCL på varje nod.
	\item Räkna ut spänningen genom att räkna ut spänningsdifferensen mellan två noder.
\end{enumerate}

Tips: Räkna $ I_{out} $ som positiv i varje resistor. 

\subsubsection*{Supernod}
Spänningskälla som ej är direkt kopplad till referensnoden kan göras om till en supernod. Nodens spänning är källans spänning och båda ändars kopplingar räknas som supernodens kopplingar.

\subsubsection*{Mesh current analysis}
Analysera loopar i en krets (medsols). Applicera KVL på varje loop. Lös med ekvationssystem.

\subsubsection*{Supermesh}
Strömkälla i kretsen. Kombinera loopar in i en större superloop. $ I_{super} = I_1 - I_2 $

\subsubsection*{Superposition}
Går endast att applicera på linjära kretsar med flera ström- och/eller spänningskällor. Varje källa kan analyseras separat för att sedan läggas ihop.
\begin{enumerate}
	\item Stäng av alla källor förutom en.
	\begin{itemize}
		\item v = 0 blir en kortsluten krets.
		\item I = 0 blir en öppen krets.
		\item Räkna ut källans kretspåverkan.
	\end{itemize}
	\item Lägg ihop alla källors påverkan.
\end{enumerate}




\end{document}